%%
% The BIThesis Template for Bachelor Graduation Thesis
%
% 北京理工大学毕业设计(论文)结论 —— 使用 XeLaTeX 编译
%
% Copyright 2020-2023 BITNP
%
% This work may be distributed and/or modified under the
% conditions of the LaTeX Project Public License, either version 1.3
% of this license or (at your option) any later version.
% The latest version of this license is in
%   http://www.latex-project.org/lppl.txt
% and version 1.3 or later is part of all distributions of LaTeX
% version 2005/12/01 or later.
%
% This work has the LPPL maintenance status `maintained'.
%
% The Current Maintainer of this work is Feng Kaiyu.
%
% Compile with: xelatex -> biber -> xelatex -> xelatex

\begin{conclusion}
\section{总结}
{对比其他的基于生物特征识别的用户认证方法,本方法对于设备具有最低的硬件要求,即拥有摄像头和闪光灯。系统通过指尖的反射光线获取指尖按压的视频流,提取心波在三个颜色通道上的振幅变化,再通过高通与带通滤波消除呼吸颤抖等因素的影响,最后通过PCA转换抑制心波小幅变化并降维简化系统,提取心脏的收缩舒张特征与非基准特征。在系统实施过程中,采用客户端-服务器的方式,前端手机应用负责采集数据并上传,后端分析并返回。在系统部署前需要采集合法用户的70个心波生成配置文件用于心脏生物特征的匹配。生成配置文件后仍需合法用户与非法用户的数据进行阈值优化和周期选择,以达到最好的验证效果,之后即可实施系统。对于前端设备只需拥有摄像头和闪光灯,大多数移动设备均具备部署条件,不需要像指纹识别一样的专用生物特征传感器,可部署设备范围广,不需要额外成本。}
\par
{对比在不同环境下的实验,该方法在较低的帧率(30fps)下仍具有较高的96.23\%的准确率,具有较高的健壮性。本文选择了帧率为30fps和60fps的设备配置,并对720p、1080p和4k三种不同的分辨率进行对比,结果表明分辨率越高效果越好,在1080p的分辨率下已经具有较好的效果,而帧率对于系统性能与安全性无太大影响,在低帧率下也能取得较好效果。对于不同手指进行在帧率为30fps,分辨率为1080p的配置下进行交叉测试,结果表明使用不同手指测试系统性能和安全性有较大幅度降低,暂不具备兼容性。就系统开销而言,由于该系统在较低帧率的分辨率下也能满足正常功能,对帧率和分辨率没有太高要求,对内存的占用和能耗开销较小,是一个轻量级的系统。}
{}
\section{未来工作}
{对比其他验证方法,比如人脸识别和指纹识别来说,由于验证时至少需要2个心动周期才能保证系统的安全性,通过心脏生物特征来验证会有一定延迟。一般人的一个心动周期为0.8秒,在不同环境或情绪状态下心率会有所不同,不同用户心率也会有所差异,用户验证时所需时间也不尽相同。加上图像获取与程序处理时间,该方法验证所需时间明显长于其他方法,为了优化系统性能,未来的研究可以在特征提取和图像获取上进行深入,在保证系统安全性的前提下,减少验证的心动周期数和图像处理时间。}
\par
{对于系统安全性而言,对比指纹识别和人脸识别来说,人的心脏特征比指纹和人脸特征更难窃取,但是在极端情况下,比如攻击者通过脉搏和血氧饱和度等方法获取合法用户的心波,或者录制合法用户的指尖按压视频进行攻击。未来的研究可以通过变换光源来来改变三色通道的心脏运动模式,让攻击者难以复现验证时的光线环境来达到拒绝欺骗验证的目的。}
\par
{对于本方法的适用范围而言,本实验选取的实验对象数量少,无法验证在大量人群中通过该方法提取的心脏生物特征是否具有独特性,未来的研究可以深入到大规模人群进行,使得提取心脏生物特征的方法在大量人群中依然具有普遍性。}
\par
{此外,本实验测试人群均为身体健康的青年,对于本身具有心脏疾病和其他年龄段的人群是否适用,需要未来的继续研究。}
\end{conclusion}

